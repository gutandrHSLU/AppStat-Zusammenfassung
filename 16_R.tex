% \section{R}
% \noindent\rule[\linienAbstand]{\linewidth}{\linienDickeDick}

\section{Basics}
\noindent\rule[\linienAbstand]{\linewidth}{\linienDicke}

\subsection{Normal Distribution}
\noindent\rule[\linienAbstand]{\linewidth}{\linienDicke}
\textbf{Normal Distribution Function \texttt{pnorm}}\\
Probability of $X$ being less than 8.10 with $\texttt{mean = 8.00}$ and $\texttt{sd = 0.05}$
\begin{equation*}
  P(X \leq 8.10) = \texttt{pnorm(q=8.10, mean=8.00, sd=0.05)} = 0.97725
\end{equation*}


\textbf{Quantile Function \texttt{qnorm}}\\
Value of $x$ for which $P(X \leq x) = 0.97725$.
\begin{equation*}
  \texttt{qnorm(p=0.97725, mean=8.00, sd=0.05)} = 8.10
\end{equation*}

\subsection{Binomial Distribution}
\noindent\rule[\linienAbstand]{\linewidth}{\linienDicke}
An experiment consists of five independent coin throws with an asymmetrical coin. For every roll, head is likely to top with the probability 0.53.\\
\textbf{Density Function \texttt{dbinom}}\\
What is the probability of throwing exactly two times head?
\begin{equation*}
  P(X=2) = \texttt{dbinom(x=2, size=5, prob=0.53)} = 0.2916388
\end{equation*}

\textbf{Distribution Function \texttt{pbinom}}\\
What is the probability of throwing a maximum of three times head?
\begin{equation*}
  P(X \leq 3) = \texttt{pbinom(q=3, size=5, prob=0.53)} = 0.7727541
\end{equation*}


\section{Utility function for finding sampling plans}
\noindent\rule[\linienAbstand]{\linewidth}{\linienDicke}
\textbf{Description}\\
Find the sampling plan with smallest sample size (single sample only) such that specified Producer Risk Point (PRP) and Consumer Risk Point (CRP) are both met.

\textbf{Arguments}\\
\begin{table}[H]
  \scriptsize
  \begin{tabularx}{\linewidth}{lX}
    \texttt{PRP} &
    The Producer Risk Point in the form of a two element numeric vector of the form \texttt{c(pd, pa)}. The first element, \texttt{pd}, specifies the quality level at which to evaluate the plan. The second element, \texttt{pa}, indicates the minimum probability of acceptance to be achieved by the plan.\\
    \texttt{CRP} &
    The Consumer Risk Point in the form of a two element numeric vector of the form \texttt{c(pd, pa)}. The first element, \texttt{pd}, specifies the quality level at which to evaluate the plan. The second element, \texttt{pa}, indicates the maximum probability of acceptance to be achieved by the plan.\\
    \texttt{type} &
    The distribution which the sampling plan is based on. Possible values are \texttt{binomial, hypergeom, poisson and normal}.\\
    \texttt{N} &
    The size of the population from which samples are drawn. Only applicable for \texttt{type="hypergeom"}.\\
    \texttt{s.type} &
    The type of the standard deviation. A value of known results in a sampling plan based on the population standard deviation, while a value of \texttt{unknown} results in the use of the sample standard deviation.
  \end{tabularx}
\end{table}

\textbf{Usage}\\
\texttt{find.plan(PRP, CRP, type="binomial")}\\
\texttt{find.plan(PRP, CRP, type="hypergeom", N)}\\
\texttt{find.plan(PRP, CRP, type="normal", s.type="unknown")}\\

\textbf{Example}\\
We want to find the smallest sample size $n$ and the acceptence number $c$. The line\\
\texttt{find.plan(PRP=c(p.alpha, 1-alpha), CRP=c(p.beta, beta), type="hypergeom", N=100)}\\
yields values for $n$ and $c$.\\
\texttt{p.alpha} and \texttt{p.beta} are the probabilities of acceptance and rejection and \texttt{alpha} and \texttt{beta} are the producer and consumer risk.\\

The real producers risk can then be found with\\
\texttt{1-func.OC(p=p.alpha,N,n=n,c=c)}\\
and the real consumers risk with\\
\texttt{func.OC(p=p.beta,N,n=n,c=c)}.\\
Where \texttt{func.OC} is the function of the operating characteristic (eq. \ref{eq:OC}).


\section{Linear Regression}
\noindent\rule[\linienAbstand]{\linewidth}{\linienDicke}
\textbf{Description}\\
\texttt{lm} is used to fit linear models. It can be used to carry out regression, single stratum analysis of variance and analysis of covariance (although \texttt{aov} may provide a more convenient interface for these).\\

\textbf{Usage}\\
\texttt{lm(formula, data, ...)}\\

\textbf{Arguments}\\
\begin{table}[H]
  \scriptsize
  \begin{tabularx}{\linewidth}{lX}
    \texttt{formula} & an object of class \texttt{"formula"}: a symbolic description of the model to be fitted. The details of model specification are given under \emph{Details}.\\
    \texttt{data} & an optional data frame, list or environment containing the variables in the model.
  \end{tabularx}
\end{table}
\textbf{Details}\\
Models for \texttt{lm} are specified symbolically. A typical model has the form \texttt{response $\sim$ terms} where \texttt{response} is the (numeric) response vector and \texttt{terms} is a series of terms which specifies a linear predictor for \texttt{response}. A terms specification of the form \texttt{first + second} indicates all the terms in \texttt{first} together with all the terms in \texttt{second} with duplicates removed. A specification of the form \texttt{first:second} indicates the set of terms obtained by taking the interactions of all terms in \texttt{first} with all terms in \texttt{second}. The specification \texttt{first*second} indicates the cross of \texttt{first} and \texttt{second}. This is the same as \texttt{first + second + first:second}.\\

A formula has an implied intercept term. To remove this use either \texttt{y $\sim$ x - 1} or \texttt{y $\sim$ 0 + x}.\\

\textbf{Value}\\
\texttt{lm} returns an object of class \texttt{lm} or for multiple responses of class \texttt{c("mlm", "lm")}.\\

The functions \texttt{summary} and \texttt{anova} are used to obtain and print a summary and analysis of variance table of the results.\\

\subsection{Simple Linear Regression}
\noindent\rule[\linienAbstand]{\linewidth}{\linienDicke}
We want to fit a linear model where the model has the form \texttt{Time $\sim$ Volume}. In R we do this with the following line\\
\texttt{mod <- lm(formula = Time $\sim$ Volume, data = data)}\\

With the following line we get a summary of the fited model:\\
\begingroup
\scriptsize
\begin{verbatim}
  > summary(mod)

  Call:
  lm(formula = Time ~ Volume, data = data)

  Residuals:
      Min      1Q  Median      3Q     Max
  -7.5811 -1.8739 -0.3493  2.1807 10.6342

  Coefficients:
              Estimate Std. Error t value Pr(>|t|)
  (Intercept)    3.321      1.371   2.422   0.0237 *
  Volume         2.176      0.124  17.546 8.22e-15 ***
  ---
  Signif. codes:  0 ‘***’ 0.001 ‘**’ 0.01 ‘*’ 0.05 ‘.’ 0.1 ‘ ’ 1

  Residual standard error: 4.181 on 23 degrees of freedom
  Multiple R-squared:  0.9305,	Adjusted R-squared:  0.9275
  F-statistic: 307.8 on 1 and 23 DF,  p-value: 8.22e-15
\end{verbatim}
\endgroup
\vspace{\baselineskip}

\subsection{Confidence Interval on the Slope}
\noindent\rule[\linienAbstand]{\linewidth}{\linienDicke}
\begingroup
\scriptsize
\begin{verbatim}
  # 95% confidence interval on slope (lm)
  > confint(mod, level=0.95)
                  2.5 %   97.5 %
  (Intercept) 0.4844979 6.157062
  Volume      1.9195920 2.432741
\end{verbatim}
\endgroup
\vspace{\baselineskip}

\subsection{Confidence Interval on the Response}
\noindent\rule[\linienAbstand]{\linewidth}{\linienDicke}
We construct 95\% confidence intervals on the response with the following line:
\begingroup
\scriptsize
\begin{verbatim}
  > Time.Conf <- predict(mod, newdata=data.new, interval="confidence",
                        level=0.95)
  > head(Time.Conf)
          fit       lwr       upr
  1 -18.44089 -23.55568 -13.32610
  2 -17.35280 -22.34705 -12.35855
  3 -16.26472 -21.13883 -11.39061
  4 -15.17664 -19.93103 -10.42224
  5 -14.08855 -18.72369  -9.45342
  6 -13.00047 -17.51684  -8.48410

\end{verbatim}
\endgroup
Where \texttt{data.new} is an array with equally spaced x-values covering at least the range of the Volume-data.\\

\subsection{Prediction Interval on the Response}
\noindent\rule[\linienAbstand]{\linewidth}{\linienDicke}
We construct 95\% prediction intervals with the following lines:
\begingroup
\scriptsize
\begin{verbatim}
  > Time.Pred <- predict(mod, newdata=data.new, interval="prediction",
                        level=0.95)
  > head(Time.Pred)
          fit       lwr       upr
  1 -18.44089 -28.48984 -8.391934
  2 -17.35280 -27.34094 -7.364664
  3 -16.26472 -26.19333 -6.336109
  4 -15.17664 -25.04703 -5.306245
  5 -14.08855 -23.90206 -4.275050
  6 -13.00047 -22.75844 -3.242499

\end{verbatim}
\endgroup

\subsection{Correlation}
\noindent\rule[\linienAbstand]{\linewidth}{\linienDicke}

\subsection{Multiple Linear Regression}
\noindent\rule[\linienAbstand]{\linewidth}{\linienDicke}
We want to fit a multiple linear regression model where the model has the form \texttt{Vibration.log $\sim$ Distance.log + Charge.log}. In R we do this with the following line:\\
\texttt{mod <- lm(Vibration.log $\sim$ Distance.log + Charge.log, data)}\\
The summary of that model is the following:
\begingroup
\scriptsize
\begin{verbatim}
  > summary(mod)

  Call:
  lm(formula = Vibration.log ~ Distance.log + Charge.log, data = data)

  Residuals:
       Min       1Q   Median       3Q      Max
  -0.43571 -0.11580  0.00227  0.09715  0.36978

  Coefficients:
               Estimate Std. Error t value Pr(>|t|)
  (Intercept)    2.8323     0.2229  12.707   <2e-16 ***
  Distance.log  -1.5107     0.1111 -13.592   <2e-16 ***
  Charge.log     0.8083     0.3042   2.658   0.0109 *
  ---
  Signif. codes:  0 ‘***’ 0.001 ‘**’ 0.01 ‘*’ 0.05 ‘.’ 0.1 ‘ ’ 1

\end{verbatim}
\endgroup

\subsection{Factor Variables}
\noindent\rule[\linienAbstand]{\linewidth}{\linienDicke}
If we want to use factor variables in a linear regression model we have do define them as such. The following lines tell R to use the position as a factor variable
\begingroup
\scriptsize
\begin{verbatim}
  # define Position as a factor
  > data$Position.fac <- as.factor(data$Position)
\end{verbatim}
\endgroup

The summary of the model is the following
\begingroup
\scriptsize
\begin{verbatim}
  > #   estimation of the parameters
  > mod.fac <- lm(Vibration.log ~ Distance.log + Charge.log
                  + Position.fac, data)
  > summary(mod.fac)

  Call:
  lm(formula = Vibration.log ~ Distance.log + Charge.log + Position.fac,
      data = data)

  Residuals:
       Min       1Q   Median       3Q      Max
  -0.36732 -0.09607  0.01385  0.08883  0.28017

  Coefficients:
                Estimate Std. Error t value Pr(>|t|)
  (Intercept)    2.51044    0.28215   8.898 3.25e-11 ***
  Distance.log  -1.33779    0.14073  -9.506 4.97e-12 ***
  Charge.log     0.69179    0.29666   2.332   0.0246 *
  Position.fac2  0.16430    0.07494   2.192   0.0340 *
  Position.fac3  0.02170    0.06366   0.341   0.7349
  Position.fac4  0.11080    0.07477   1.482   0.1459
  ---
  Signif. codes:  0 ‘***’ 0.001 ‘**’ 0.01 ‘*’ 0.05 ‘.’ 0.1 ‘ ’ 1

\end{verbatim}
\endgroup

We need to understand how to interpret the estimated coefficients of the factor levels:
For every position we get a different estimated model. These four models have the slope $\hat{\beta}_1 = -1.33779$ (coefficient of \texttt{Distance.log}) and the slope $\hat{\beta}_2 = 0.69179$ (coefficient of \texttt{Charge.log}) in common.
But they all have different intercepts. The model at the first position has the intercept $\hat{\beta}_0 = 2.51044$ (reference model due to our constraint), the model at the second position has the intercept $\hat{\beta}_0 + \hat{\beta}_{3,2} = 2.51044 + 0.16430 = 2.67474$, the model at the third position has the intercept $\hat{\beta}_0 + \hat{\beta}_{3,3} = 2.51044 + 0.02170 = 2.53214$ and the model at the forth position has the intercept $\hat{\beta}_0 + \hat{\beta}_{3,4} = 2.51044 + 0.11080 = 2.62124$.\\

The $t$-statistic and the $P$-value of the factor levels have little meaning. With our constraint $\beta_{3,1} = 0$ they simply show if the difference between the corresponding positions and the first (reference) position is significant.


\subsection{F-test to compare two models}
\noindent\rule[\linienAbstand]{\linewidth}{\linienDicke}
We come back to the explosion example and want to ask the question whether the factor \texttt{Position.fac} has a significant influence on the vibrations.\\

\textbf{ANOVA for linear model fits}\\
\begingroup
\scriptsize
\begin{verbatim}
  > anova(mod.fac, mod)
  Analysis of Variance Table

  Model 1: Vibration.log ~ Distance.log + Charge.log + Position.fac
  Model 2: Vibration.log ~ Distance.log + Charge.log
    Res.Df     RSS Df Sum of Sq      F  Pr(>F)
  1     42 0.90467
  2     45 1.05217 -3   -0.1475 2.2827 0.09297 .
  ---
  Signif. codes:  0 ‘***’ 0.001 ‘**’ 0.01 ‘*’ 0.05 ‘.’ 0.1 ‘ ’ 1

\end{verbatim}
\endgroup
In the R-output we find $F = 2.2827$ with a $P$-value of $0.09297$. Therefore the influence of the factor \texttt{Position.fac} is not significant on the $5\%$ level.\\

\textbf{F-test with \texttt{drop1} command}\\
An other, more powerful, R-command to perform an $F$-test to compare two models is \texttt{drop1}, which drops all possible single terms to a model, fits those models and computes a table of the changes in fit.\\

\begingroup
\scriptsize
\begin{verbatim}
  > drop1(mod.fac, test="F")
  Single term deletions

  Model:
  Vibration.log ~ Distance.log + Charge.log + Position.fac
               Df Sum of Sq     RSS     AIC F value    Pr(>F)
  <none>                    0.90467 -178.63
  Distance.log  1   1.94659 2.85125 -125.53 90.3722 4.973e-12 ***
  Charge.log    1   0.11713 1.02180 -174.78  5.4379   0.02457 *
  Position.fac  3   0.14750 1.05217 -177.38  2.2827   0.09297 .
  ---
  Signif. codes:  0 ‘***’ 0.001 ‘**’ 0.01 ‘*’ 0.05 ‘.’ 0.1 ‘ ’ 1

\end{verbatim}
\endgroup
In the last column we find the $P$-values for the variables \texttt{Distance.log} and \texttt{Charge.log}. These two variables are significant on the $5\%$ level. We find also the $P$-value of $0.09297$ for the factor \texttt{Position.fac}. Therefore its influence is not significant on the $5\%$ level.\\


\subsection{Model Selection}
\noindent\rule[\linienAbstand]{\linewidth}{\linienDicke}
In R we can carry out variable selection based on Akaike information criterion. First we fit the trivial model with only an intercept as the start of the search algorithm. \texttt{> mod.small <- lm(C.log ~ 1, data)}\\
Any other model would also work.\\
We use the R-function \texttt{step()} which chooses a model by AIC in a stepwise algorithm. With the argument \texttt{scope} we specify the smallest and the largest model between which the most suitable model should be found. In addition we specify the direction of our search algorithm.\\

\begingroup
  \scriptsize
  \begin{verbatim}
  > mod.both <- step(mod.small, scope=list(upper= ~ D + T1 + T2
  +                   + S.log + PR + NE + CT + BW + N.sqrt + PT,
  +                   lower= ~ 1),
  +                   direction="both")

  \end{verbatim}
\endgroup
According to the stepwise variable selection in both directions based on the AIC criterion the following model results.\\

\begingroup
  \scriptsize
  \begin{verbatim}
  summary(mod.both)
  Call:
  lm(formula = C.log ~ PT + S.log + D + NE + CT + N.sqrt, data = data)
  Residuals:
       Min       1Q   Median       3Q      Max
  -0.14982 -0.03014  0.01074  0.03817  0.14014
  Coefficients:
              Estimate Std. Error t value Pr(>|t|)
  (Intercept) -5.82981    1.48707  -3.920 0.000608 ***
  PT          -0.10441    0.05038  -2.072 0.048700 *
  S.log        0.72559    0.12218   5.939 3.37e-06 ***
  D            0.09341    0.02053   4.551 0.000119 ***
  NE           0.10449    0.03263   3.202 0.003694 **
  CT           0.06523    0.02708   2.409 0.023707 *
  N.sqrt      -0.03046    0.01646  -1.850 0.076133 .
  ---
  Signif. codes:  0 '***' 0.001 '**' 0.01 '*' 0.05 '.' 0.1 ' ' 1

\end{verbatim}
\endgroup


\subsection{Collinearity}
\noindent\rule[\linienAbstand]{\linewidth}{\linienDicke}
We calculate the variance inflation factor for all explanatory variables used in an optimal model.\\

\begingroup
  \scriptsize
  \begin{verbatim}
  # fit optimal model
  > mod.opt <- lm(C.log ~ PT + S.log + D + NE + CT + N.sqrt, data)
  #   variance inflation factor
  > vif(mod.opt)
        PT    S.log        D       NE       CT   N.sqrt
  2.497443 1.089288 2.716501 1.289015 1.142437 2.248181
  \end{verbatim}
\endgroup
All variance inflation factors are smaller than 5. Therefore there is no problem with collinearity in the optimal model\\

\subsection{Analysis of Variance}
\noindent\rule[\linienAbstand]{\linewidth}{\linienDicke}
\begingroup
  \scriptsize
  \begin{verbatim}

  \end{verbatim}
\endgroup


\textbf{Description}\\
Fit an analysis of variance model by a call to lm for each stratum.\\

\textbf{Usage}\\
\texttt{aov(formula, data, ...)}\\

\textbf{Arguments}\\
\begin{table}[H]
  \scriptsize
  \begin{tabularx}{\linewidth}{lX}
    \texttt{formula} & A formula specifying the model. (Analogus to the \texttt{lm} function).\\
    \texttt{data} & A data frame in which the variables specified in the formula will be found.
  \end{tabularx}
\end{table}

\textbf{Example}\\
\begingroup
\scriptsize
\begin{verbatim}
  # analysis of variance table for two factors (aov)
  > mod <- aov(Strength ~ Method + Batch, data)
  > summary(mod)
  Df Sum Sq Mean Sq F value Pr(>F)
  Method 2 23.23 11.614 8.695 0.002278 **
  Batch 9 86.79 9.644 7.219 0.000202 ***
  Residuals 18 24.04 1.336
  ---
  Signif. codes: 0 ’***’ 0.001 ’**’ 0.01 ’*’ 0.05 ’.’ 0.1 ’ ’ 1

\end{verbatim}
\endgroup
A low P-value means that the effect of, say the \texttt{Batch}-effect, is significant.\\
For the result to be valid, the following must be fulfilled: The errors must be independent and normally distributed with expected value $0$ and constant variance $\sigma^2$. Furthermore, the two factors must overlap additively. With the interaction plot (\texttt{interaction.plot()}) it can be checked graphically whether the factor effects overlap additively. If they overlap additively, the lines in the interaction plot must be parallel except for random deviations.

% \begingroup
%   \scriptsize
%   \begin{verbatim}
%
%   \end{verbatim}
% \endgroup

% \subsection{}
% \noindent\rule[\linienAbstand]{\linewidth}{\linienDicke}
% \textbf{Description}\\
% \textbf{Usage}\\
% \textbf{Arguments}\\
% \textbf{Details}\\
% \textbf{Example}\\
