% \section{R}
% \noindent\rule[\linienAbstand]{\linewidth}{\linienDickeDick}

\section{Basics}
\noindent\rule[\linienAbstand]{\linewidth}{\linienDicke}
\textbf{Normal Distribution Function \texttt{pnorm}}\\
Probability of $X$ being less than 8.10 with $\texttt{mean = 8.00}$ and $\texttt{sd = 0.05}$
\begin{equation*}
  \begin{split}
    P(X \leq 8.10) &= \texttt{pnorm(q=8.10, mean=8.00, sd=0.05)}\\
    &= 0.97725
  \end{split}
\end{equation*}

\textbf{Quantile Function \texttt{qnorm}}\\
Value of $x$ for which $P(X \leq x) = 0.97725$.
\begin{equation*}
  \texttt{qnorm(p=0.97725, mean=8.00, sd=0.05)} = 8.10
\end{equation*}

\section{Utility function for finding sampling plans}
\noindent\rule[\linienAbstand]{\linewidth}{\linienDicke}
\textbf{Description}\\
Find the sampling plan with smallest sample size (single sample only) such that specified Producer Risk Point (PRP) and Consumer Risk Point (CRP) are both met.

\textbf{Arguments}\\
\begin{table}[H]
  \scriptsize
  \begin{tabularx}{\linewidth}{lX}
    \texttt{PRP} &
    The Producer Risk Point in the form of a two element numeric vector of the form \texttt{c(pd, pa)}. The first element, \texttt{pd}, specifies the quality level at which to evaluate the plan. The second element, \texttt{pa}, indicates the minimum probability of acceptance to be achieved by the plan.\\
    \texttt{CRP} &
    The Consumer Risk Point in the form of a two element numeric vector of the form \texttt{c(pd, pa)}. The first element, \texttt{pd}, specifies the quality level at which to evaluate the plan. The second element, \texttt{pa}, indicates the maximum probability of acceptance to be achieved by the plan.\\
    \texttt{type} &
    The distribution which the sampling plan is based on. Possible values are \texttt{binomial, hypergeom, poisson and normal}.\\
    \texttt{N} &
    The size of the population from which samples are drawn. Only applicable for \texttt{type="hypergeom"}.\\
    \texttt{s.type} &
    The type of the standard deviation. A value of known results in a sampling plan based on the population standard deviation, while a value of \texttt{unknown} results in the use of the sample standard deviation.
  \end{tabularx}
\end{table}

\textbf{Usage}\\
\texttt{find.plan(PRP, CRP, type="binomial")}\\
\texttt{find.plan(PRP, CRP, type="hypergeom", N)}\\
\texttt{find.plan(PRP, CRP, type="normal", s.type="unknown")}\\

\textbf{Example}\\
We want to find the smallest sample size $n$ and the acceptence number $c$. The line\\
\texttt{find.plan(PRP=c(p.alpha, 1-alpha), CRP=c(p.beta, beta), type="hypergeom", N=100)}\\
yields values for $n$ and $c$.\\
\texttt{p.alpha} and \texttt{p.beta} are the probabilities of acceptance and rejection and \texttt{alpha} and \texttt{beta} are the producer and consumer risk.\\

The real producers risk can then be found with\\
\texttt{1-func.OC(p=p.alpha,N,n=n,c=c)}\\
and the real consumers risk with\\
\texttt{func.OC(p=p.beta,N,n=n,c=c)}.\\
Where \texttt{func.OC} is the function of the operating characteristic (eq. \ref{eq:OC}).


\section{Multiple Regression}
\noindent\rule[\linienAbstand]{\linewidth}{\linienDicke}
\textbf{Description}\\
\texttt{lm} is used to fit linear models. It can be used to carry out regression, single stratum analysis of variance and analysis of covariance (although \texttt{aov} may provide a more convenient interface for these).\\

\textbf{Usage}\\
\texttt{lm(formula, data, ...)}\\

\textbf{Arguments}\\
\begin{table}[H]
  \scriptsize
  \begin{tabularx}{\linewidth}{lX}
    \texttt{formula} & an object of class \texttt{"formula"}: a symbolic description of the model to be fitted. The details of model specification are given under \emph{Details}.\\
    \texttt{data} & an optional data frame, list or environment containing the variables in the model.
  \end{tabularx}
\end{table}
\textbf{Details}\\
Models for \texttt{lm} are specified symbolically. A typical model has the form \texttt{response $\sim$ terms} where \texttt{response} is the (numeric) response vector and \texttt{terms} is a series of terms which specifies a linear predictor for \texttt{response}. A terms specification of the form \texttt{first + second} indicates all the terms in \texttt{first} together with all the terms in \texttt{second} with duplicates removed. A specification of the form \texttt{first:second} indicates the set of terms obtained by taking the interactions of all terms in \texttt{first} with all terms in \texttt{second}. The specification \texttt{first*second} indicates the cross of \texttt{first} and \texttt{second}. This is the same as \texttt{first + second + first:second}.\\

A formula has an implied intercept term. To remove this use either \texttt{y $\sim$ x - 1} or \texttt{y $\sim$ 0 + x}.\\

\textbf{Value}\\
\texttt{lm} returns an object of class \texttt{lm} or for multiple responses of class \texttt{c("mlm", "lm")}.\\

The functions \texttt{summary} and \texttt{anova} are used to obtain and print a summary and analysis of variance table of the results.\\

\subsection{Simple Linear Regression}
\noindent\rule[\linienAbstand]{\linewidth}{\linienDicke}
We want to fit a linear model where the model has the form \texttt{Time $\sim$ Volume}. In R we do this with the following line\\
\texttt{mod <- lm(formula = Time $\sim$ Volume, data = data)}\\

With the following line we get a summary of the fited model:\\
\begingroup
\scriptsize
\begin{verbatim}
> summary(mod)

Call:
lm(formula = Time ~ Volume, data = data)

Residuals:
    Min      1Q  Median      3Q     Max
-7.5811 -1.8739 -0.3493  2.1807 10.6342

Coefficients:
            Estimate Std. Error t value Pr(>|t|)
(Intercept)    3.321      1.371   2.422   0.0237 *
Volume         2.176      0.124  17.546 8.22e-15 ***
---
Signif. codes:  0 ‘***’ 0.001 ‘**’ 0.01 ‘*’ 0.05 ‘.’ 0.1 ‘ ’ 1

Residual standard error: 4.181 on 23 degrees of freedom
Multiple R-squared:  0.9305,	Adjusted R-squared:  0.9275
F-statistic: 307.8 on 1 and 23 DF,  p-value: 8.22e-15
\end{verbatim}
\endgroup
\vspace{\baselineskip}

\subsection{Confidence Interval on Slope}
\noindent\rule[\linienAbstand]{\linewidth}{\linienDicke}
\begingroup
\scriptsize
\begin{verbatim}
# 95% confidence interval on slope (lm)
> confint(mod, level=0.95)
                2.5 %   97.5 %
(Intercept) 0.4844979 6.157062
Volume      1.9195920 2.432741
\end{verbatim}
\endgroup
\vspace{\baselineskip}

\subsection{Confidence Interval on the Response}
\noindent\rule[\linienAbstand]{\linewidth}{\linienDicke}
We construct 95\% confidence intervals on the response with the following line:
\begingroup
\scriptsize
\begin{verbatim}
> Time.Conf <- predict(mod, newdata=data.new, interval="confidence",
                      level=0.95)
> head(Time.Conf)
        fit       lwr       upr
1 -18.44089 -23.55568 -13.32610
2 -17.35280 -22.34705 -12.35855
3 -16.26472 -21.13883 -11.39061
4 -15.17664 -19.93103 -10.42224
5 -14.08855 -18.72369  -9.45342
6 -13.00047 -17.51684  -8.48410
\end{verbatim}
\endgroup
Where \texttt{data.new} is an array with equally spaced x-values covering at least the range of the Volume-data.\\

\subsection{Prediction Interval on the Response}
\noindent\rule[\linienAbstand]{\linewidth}{\linienDicke}
We construct 95\% prediction intervals with the following lines:
\begingroup
\scriptsize
\begin{verbatim}
> Time.Pred <- predict(mod, newdata=data.new, interval="prediction",
                      level=0.95)
> head(Time.Pred)
        fit       lwr       upr
1 -18.44089 -28.48984 -8.391934
2 -17.35280 -27.34094 -7.364664
3 -16.26472 -26.19333 -6.336109
4 -15.17664 -25.04703 -5.306245
5 -14.08855 -23.90206 -4.275050
6 -13.00047 -22.75844 -3.242499
\end{verbatim}
\endgroup

\subsection{Correlation}
\noindent\rule[\linienAbstand]{\linewidth}{\linienDicke}

\subsection{Multiple Linear Regression}
\noindent\rule[\linienAbstand]{\linewidth}{\linienDicke}
We want to fit a multiple linear regression model where the model has the form \texttt{Vibration.log $\sim$ Distance.log + Charge.log}. In R we do this with the following line:
\texttt{mod <- lm(Vibration.log $\sim$ Distance.log + Charge.log, data)}
The summary of that model is the following:
\begingroup
\scriptsize
\begin{verbatim}
> summary(mod)

Call:
lm(formula = Vibration.log ~ Distance.log + Charge.log, data = data)

Residuals:
     Min       1Q   Median       3Q      Max
-0.43571 -0.11580  0.00227  0.09715  0.36978

Coefficients:
             Estimate Std. Error t value Pr(>|t|)
(Intercept)    2.8323     0.2229  12.707   <2e-16 ***
Distance.log  -1.5107     0.1111 -13.592   <2e-16 ***
Charge.log     0.8083     0.3042   2.658   0.0109 *
---
Signif. codes:  0 ‘***’ 0.001 ‘**’ 0.01 ‘*’ 0.05 ‘.’ 0.1 ‘ ’ 1

Residual standard error: 0.1529 on 45 degrees of freedom
Multiple R-squared:  0.8048,	Adjusted R-squared:  0.7962
F-statistic: 92.79 on 2 and 45 DF,  p-value: < 2.2e-16
\end{verbatim}
\endgroup

\subsection{Binary Explanatory Variables}
\noindent\rule[\linienAbstand]{\linewidth}{\linienDicke}
\subsection{Factor Variables}
\noindent\rule[\linienAbstand]{\linewidth}{\linienDicke}

% \begingroup
%   \scriptsize
%   \begin{verbatim}
%
%   \end{verbatim}
% \endgroup

% \subsection{}
% \noindent\rule[\linienAbstand]{\linewidth}{\linienDicke}
% \textbf{Description}\\
% \textbf{Usage}\\
% \textbf{Arguments}\\
% \textbf{Details}\\
% \textbf{Example}\\
