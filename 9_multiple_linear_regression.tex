\section{Multiple Linear Regression}
\noindent\rule[\linienAbstand]{\linewidth}{\linienDickeDick}
A multiple regression model is a model which involves more than one regressor variable.\\

\subsection{Model and Estimation}
\noindent\rule[\linienAbstand]{\linewidth}{\linienDicke}
We generalise the concept of a simple linear regression model with only one explanatory variable to a multiple linear regression model
\begin{equation}
  y = \beta_0 + \beta_1 x_1 + \beta_2 x_2 + ... + \beta_m x_m +\varepsilon,
\end{equation}

$\beta_0$ is the intercept\\
$\beta_k$ is the slope in the $x_k$ direction for all $k \in {1, 2, ..., m}$.


\textbf{Least-Squares Estimation of the Regression Coefficient}\\
The general idea and philosophy of teh multiple linear regression modle are the same as in the simple linear regression model. To calculate the according factors the matrix notation is used.
\begin{equation}
  \mathbf{y} = \mathbf{X} \mathbf{\beta} + \varepsilon
\end{equation}

Aufg. 8.5.1 is important!!

\subsection{Diversity of Modelling Possibilities}
\noindent\rule[\linienAbstand]{\linewidth}{\linienDicke}
Explanatory variables can be strongly (linearly) correlated. It is however difficult to interpret the results of the regression analysis and should be avoided.\\

\textbf{Polynomial Regression}\\
If we want to use a polynomial of degree two to describe a relationship, then the model has the following form
\begin{equation}
  y = \beta_0 + \beta_1x + \beta_2 x^2 + \varepsilon
\end{equation}
If we define the new variables $\texttt{lin} = x$ and $\texttt{quad} = x^2$, then we obtain mathematically the same model
\begin{equation}
  y = \beta_0 + \beta_1 \cdot \texttt{lin} + \beta_2 \cdot \texttt{quad} + \varepsilon
\end{equation}
which is now a multiple linear regression model.\\

\textbf{Nonlinear Functions and Linear Regression}\\
Experience shows that in most real-world regression problems, the variables must be transformed to get the best results. Four examples:
\begin{itemize}
  \item Inverting the nonlinear equations
  \begin{equation}
    \begin{split}
      y =& \frac{1}{a+be^{-x}}\\
      y =& \frac{ax}{b+x}
    \end{split}
    \;\;\;\;\;\;
    \begin{split}
      \text{gives}\\
      \text{gives}
    \end{split}
    \;\;\;\;\;\;
    \begin{split}
      \frac{1}{y} =& a + be^{-x}\\
      \frac{1}{y} =& \frac{1}{a} + \frac{a}{b} \frac{1}{x}
    \end{split}
  \end{equation}
  \item Taking logarithms of the nonlinear equations
  \begin{equation}
    \begin{split}
        y &= ax^b\\
        y &= ax^{bg(x)}
    \end{split}
    \;\;\;\;\;\;
    \begin{split}
      \text{gives}\\
      \text{gives}
    \end{split}
    \;\;\;\;\;\;
    \begin{split}
      \textup{ln}(y) &= \textup{ln}(a) + b\textup{ln}(x)\\
      \textup{ln}(y) &= \textup{ln}(a) + bg(x)
    \end{split}
  \end{equation}
\end{itemize}
It is important to realise that also such transformations change the form of the error term. If we are not allowed to alter the original additive error term, then we need to use nonlinear least-squares regression.\\

\textbf{Binary Explanatory Variables}\\
In all our examples, the explanatory variables have been continuous so far. This is not necessary.\\
If the binary variable $x_b$ is the only variable in the model
\begin{equation}
  y = \beta_0 + \beta_1 x_b + \varepsilon
\end{equation}
then
\begin{equation}
  \begin{split}
    y =& \beta_0 + \varepsilon\\
    y =& \beta_0 + \beta_1 + \varepsilon
  \end{split}
  \;\;\;
  \begin{split}
    \textup{if}\; x_b = 0\\
    \textup{if}\; x_b = 1
  \end{split}
\end{equation}

\textbf{Factor Variables}\\
If a discrete variable has more than two levels it is called a factor.\\
But I don't get this chapter, sorry.

\subsection{Comparison of Regression Models with $F$-Test}
\noindent\rule[\linienAbstand]{\linewidth}{\linienDicke}
The $F$-test compares two models and tests whether any of the explanatory variables in a multiple linear regression model are significant.
