\section{Residual Analysis}
\noindent\rule[\linienAbstand]{\linewidth}{\linienDickeDick}


\subsection{Introduction}
\noindent\rule[\linienAbstand]{\linewidth}{\linienDicke}
In our simple linear regression models we made the following assumptions:
\begin{enumerate}
  \item The relationship between response and the regressors is linear.
  \item The error $\varepsilon$ has mean zero.
  \item The error $\varepsilon$ has constant variance $\sigma^2$
  \item The errors are uncorrelated.
  \item The errors are normally distributed
\end{enumerate}
If some assumptions are violated we should be able to see them in the errors $\varepsilon_i$. On the other hand, the errors are unknown to us, so we have to deal with the residuals
\begin{equation}
  e_i = y_i - \hat{y}_i \;\;\;\text{for all}\;\;\; i \in {1,...,n}
\end{equation}
instead. The residuals are estimators of the random errors.\\

\textbf{Scaled Residuals}\\
If the errors are normally distributed, so are the residuals of a least-squares estimate. Since the variance of the residuals depends on $x_0$ it is not equal to the variance of the errors. Therefore we use scaled residuals
\begin{equation}
  \tilde{e}_i = \frac{e_i}{\sqrt{1 - \left(\frac{1}{n} + \frac{(x_i - \bar{x})^2}{S_{xx}}\right)}} \;\;\;\text{for all}\;\;\; i \in {1,...,n}
\end{equation}


\textbf{Coefficient of Determination}\\
The coefficient of determination is a measure of the linear relationship between the response variable and the fit.\\

In simple linear regression with only one explanatory variable the coefficient of determination is defined by
\begin{equation}
  R^2 = \frac{SS_{\textup{fit}}}{S_{yy}}
\end{equation}
where
\begin{equation}
  \begin{split}
    SS_{\textup{fit}} &= \sum_{i=1}^n(\hat{y}_i - \bar{\hat{y}}_i)^2 = \hat{\beta}_1^2 S_{xx}\\
    SS_{yy} &= \sum_{i=1}^n(y_i - \bar{y}_i)^2
  \end{split}
\end{equation}
Therefore the coefficient of determination is
\begin{equation}
  R^2 = \hat{\beta}_1^2 \frac{S_{xx}}{S_{yy}} = \frac{S_{xy}^2}{S_{xx} S_{yy}} = \textup{Cor}(x, y)^2
\end{equation}

In multiple regression and simple linear regression with at least one explanatory variable the coefficient of determination is identical to the squared correlation between the response variable $y$ and the fitted values $\hat{y}$
\begin{equation}
  R^2 = \textup{Cor}(y, \hat{y})^2 = \frac{S_{y\hat{y}}^2}{S_{yy} S_{\hat{y}\hat{y}}}
\end{equation}

The coefficient of determination is a global measure for the goodness of fit and it says nothing about the suitability of the regression model.

\subsection{Diagnostic Tools}
\noindent\rule[\linienAbstand]{\linewidth}{\linienDicke}

\textbf{Tukey-Anscombe Plot}
