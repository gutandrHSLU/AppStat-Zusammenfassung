% ========== Packages ==========
\usepackage[a4paper,
  left=10mm,
  right=10mm,
  top=10mm,
  bottom=17mm
]{geometry}

% \usepackage[ngerman]{babel} %Ändert die Sprache
\usepackage[T1]{fontenc} %Wichtig für ä ö ü
\usepackage{amssymb,amsmath,amsthm,amsfonts}
\usepackage{graphicx}
\usepackage{fancyhdr}
\usepackage[utf8]{inputenc}
\usepackage{multicol,multirow}
\usepackage{longtable} %Für lange Tabellen
\usepackage{arydshln} %Für gestrichelte Linien in Tabellen
\usepackage{tabularx}
\usepackage{pdfpages} %Zum einfügen von PDF's
\usepackage{hyperref} %Für hyperlinks
\hypersetup{bookmarks=true}
\usepackage{parskip}
\usepackage{caption} %Für die Beschriftung von Bilder
\captionsetup{justification=centering}
\captionsetup{font=it}
\setlength{\parindent}{0pt}
\usepackage{subcaption} %Für die Beschriftung unterteilter Bilder
\usepackage{float}
\floatstyle{plaintop}
\restylefloat{table}
\usepackage{siunitx}%Für einheiten im Symbolverzeichnis
% \usepackage[symbols,nogroupskip,sort=none]{glossaries-extra}%Für Symbolverzeichnis
% % \input{symbolverzeichnis}
\usepackage{lipsum}%Für pseudo text



\makeatletter %Für römische Zahlen
\newcommand*{\rom}[1]{\expandafter\@slowromancap\romannumeral #1@}

%Für die dicken Linien in Tabellen
\def\thickhline{%
  \noalign{\ifnum0=`}\fi\hrule \@height \thickarrayrulewidth \futurelet
   \reserved@a\@xthickhline}
\def\@xthickhline{\ifx\reserved@a\thickhline
               \vskip\doublerulesep
               \vskip-\thickarrayrulewidth
               \fi
      \ifnum0=`{\fi}}
\makeatother
\newlength{\thickarrayrulewidth}
\setlength{\thickarrayrulewidth}{2\arrayrulewidth}

%Sorgt dafür, dass nicht immer alles auf die ganze Seite verteilt wird.
\raggedbottom

% ========== Header and Footer ==========
\pagestyle{fancy}
\fancyhf{}
% \fancyhead[RE,LO]{Seite \thepage}
% \fancyhead[LE,RO]{\nouppercase{\leftmark}}
\fancyfoot[RE,LO]{Page \thepage}
\renewcommand{\headrulewidth}{0pt}
\renewcommand{\footrulewidth}{.5pt}

%Eigens erstellte Variablen
\newcommand{\plotWidth}{0.7}
\newcommand{\garphWidth}{0.7}
\newcommand{\linienAbstand}{2ex}
\newcommand{\linienDicke}{0.5pt}
\newcommand{\linienDickeDick}{1.5pt}


% \usepackage{calc}
% \usepackage{ifthen}
% \ifthenelse{\lengthtest { \paperwidth = 11in}}
%     { \geometry{top=.5in,left=.5in,right=.5in,bottom=.5in} }
% 	{\ifthenelse{ \lengthtest{ \paperwidth = 297mm}}
% 		{\geometry{top=1cm,left=1cm,right=1cm,bottom=1cm} }
% 		{\geometry{top=1cm,left=1cm,right=1cm,bottom=1cm} }
% 	}
% \pagestyle{empty}
\makeatletter
\renewcommand{\section}{\@startsection{section}{1}{0mm}%
                                {-1ex plus -.5ex minus -.2ex}%
                                {0.5ex plus .2ex}%x
                                {\normalfont\large\bfseries}}
\renewcommand{\subsection}{\@startsection{subsection}{2}{0mm}%
                                {-1explus -.5ex minus -.2ex}%
                                {0.5ex plus .2ex}%
                                {\normalfont\normalsize\bfseries}}
\renewcommand{\subsubsection}{\@startsection{subsubsection}{3}{0mm}%
                                {-1ex plus -.5ex minus -.2ex}%
                                {1ex plus .2ex}%
                                {\normalfont\small\bfseries}}
\makeatother
\setcounter{secnumdepth}{0}
\setlength{\parindent}{0pt}
\setlength{\parskip}{0pt plus 0.5ex}
% -----------------------------------------------------------------------
