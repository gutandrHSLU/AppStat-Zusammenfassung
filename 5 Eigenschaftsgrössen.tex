\section{Definitionen und Voraussetzungen}
\noindent\rule[\linienAbstand]{\linewidth}{\linienDickeDick}

\subsection{Einteilung der Tragwerke}
\noindent\rule[\linienAbstand]{\linewidth}{\linienDicke}
\textbf{Reiner Stab und Balken}
\begin{figure}[H]
  \centering
  \includegraphics[width = 0.9\linewidth]{Pics/5.1.png}
\end{figure}
\textbf{Scheibe und Platte}
\begin{figure}[H]
  \centering
  \includegraphics[width = 1\linewidth]{Pics/5.2.png}
\end{figure}
\textbf{Schale}
\begin{figure}[H]
  \centering
  \includegraphics[width = 0.6\linewidth]{Pics/5.4.png}
\end{figure}

\subsection{Steifigkeiten}
\noindent\rule[\linienAbstand]{\linewidth}{\linienDicke}
\textbf{Widerstandsfähigkeiten}
\begin{itemize}
  \item Dehnsteifigkeit (bei Zug- oder Druckkraft),
  \item Biegesteifigkeit (bei Biegemoment),
  \item Torsionssteifigkeit (bei Torsionsmoment).\\
\end{itemize}

\textbf{Steifigkeiten}
\begin{itemize}
  \item starr (Steifigkeit $\rightarrow  \inf$, Verformung $\rightarrow 0$, Spannung endlich),
  \item steif (Steifigkeit gross, Verformung und Spannung endlich),
  \item weich (Steifigkeit klein, Verformung und Spannung endlich),
  \item schlaff (Steifigkeit $\rightarrow 0$, Verformung endlich, Spannung $\rightarrow 0$).
\end{itemize}


\subsection{Strukturelle Idealisierung}
\noindent\rule[\linienAbstand]{\linewidth}{\linienDicke}

\begin{figure}[H]
  \centering
  \includegraphics[width = 0.7\linewidth]{Pics/5.12.png}
\end{figure}

\begin{equation}
  A_{G1} = \frac{h \cdot t}{6}\cdot \left(2+ \frac{\sigma_2}{\sigma_1}\right)
\end{equation}
\begin{equation}
  A_{G2} = \frac{h \cdot t}{6}\cdot \left(2+ \frac{\sigma_1}{\sigma_2}\right)
\end{equation}



% \subsection{Leitrad}
% \noindent\rule[\linienAbstand]{\linewidth}{\linienDicke}



% \section{Querkraftschub an dünnwandigen Querschnitten}
% \noindent\rule[\linienAbstand]{\linewidth}{\linienDickeDick}
%
% \textbf{Ersatzquerkraft}
% \begin{equation}
%   Q_y = \frac{Q_y + Q_z \cdot \frac{I_{yz}}{I_y}}{1-\frac{{I_{yz}}^2}{I_y \cdot I_z}}
% \end{equation}
% \begin{equation}
%   Q_z = \frac{Q_z + Q_y \cdot \frac{I_{yz}}{I_z}}{1-\frac{{I_{yz}}^2}{I_y \cdot I_z}}
% \end{equation}
%
%
% Schubfluss
% \begin{equation}
%   q_{zwischen i und i+1} = -\frac{Q_z}{I_y} \cdot A_i \cdot z_i - \frac{Q_y}{I_z} \cdot A_i \cdot y_i
% \end{equation}
%
%
